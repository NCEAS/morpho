\section{What's new in version 2.0.0}

Morpho 2.0.0 represents a major change in how the software operates. 
Most notably, there is support for using Morpho with any 
DataONE Member Node, one of which continues to be Metacat.
These changes include:
\begin{itemize}
 \item Support for any data package identifier, not just the legacy format
 \item Client certificate-based authentication mechanism using InCommon Identity providers
 \item Mutable access control rules on a per-revision basis
 \item Mutable replication policies on a per-object-revision basis
\end{itemize}

\subsection{Identifiers}

Morpho will no longer generate data and datapackage identifiers
 in the format ``scope.id.rev''. Instead, an opaque and more universally unique
UUID will be generated. 
Previous versions of Morpho and Metacat used
a prescribed identifier format that encoded information about the object
within the identifier - namely, the revision history of the object.
Morpho now tracks this information independently from the object identifier
in the ``System Metadata'' for each object. Morpho continues to track the revision
history, but no attempt should be made to ``interpret'' any meaning from the identifiers used.

For example, a current data package with two previous revisions will look like this:
\begin{itemize}
 \item ``urn:uuid:d7440f1b-390c-4ea7-99c7-1e353e22cd23'' obsoletes ``urn:uuid:e41ba25a-6719-4dfa-b349-1ac208522647''
 \item ``urn:uuid:d7440f1b-390c-4ea7-99c7-1e353e22cd23''  obsoletedBy ``urn:uuid:05f182f9-134b-4560-8cc8-0d6ca62bc573'' 
\end{itemize}

In the future, Member Nodes may use utilize different identifier generation schemes
such as DOI, ARK, LSID, etc.

\subsection{Authentication}

Authentication will now be handled by distributed Identity Providers 
that are part of the CILogon federation. This includes InCommon providers
as well as some OpenId providers.

PLEASE NOTE: You will be using a new identity for all interactions with 
the Member Node and the DataONE network.

For existing KNB, NCEAS, and other closely-affiliated organizational accounts, the user id 
and password will still be utililized for authentication. There may be special considerations 
for user accounts that exist with multiple affiliations, but those will be converted case by 
case.

LTER-affiliated users can authenticate using their LTER Identity Provider credentials
as instructed by the network office.

All other users can either create a free ProtectNetwork account, or request an affiliated 
KNB-managed account from NCEAS.

\subsection{Access Control Policies}

Morpho now allows access control policies to be managed outside of the EML
file. Existing packages will utilize the EML-defined access control rules initially.
If modifcations are made to the access policy for the metadata object, they will be set
directly on the Coordinating Node which is responsible for broadcasting the change
throughout the network. In cases where the Member Node content has not been synchronized
to the Coordinating Node, modifications to the access policy cannot be performed.

Access control for data objects is controlled similarly. Newly imported data obejcts
will initially ``inherit'' the policy from the metadat object, but subsequent modifications
to the metadata's access policy will not cascade to the data obejcts. Access policies for 
metadata and data will be handled completely independently from that point forward.

\subsection{Replication Policies}

DataONE has introduced a granular replication policy that is applied to each 
revision of each object. These policies can be used to specify if metadata and 
data should be replicated throughout the DataONE network and, if so, how many 
replicas and on which Member Nodes those replicas should be stored. 
Changes to the replication policy of an object do not represent an ``update'' to 
the object and only serve to instruct the Coordinating Node how best to 
manage the object[s] in question.
