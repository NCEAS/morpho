\section{Introduction}

The Morpho User Guide is provided to assist scientists who want to use
the Morpho application locally or both locally and on a network to
manage, discover, and share data sets.

If you cannot find the information you are looking for in the User
Guide, please contact \href{mailto:morpho-dev@ecoinformatics.org}
{\nolinkurl{morpho-dev@ecoinformatics.org}}.

\subsection{What is Morpho?}

Created for scientists, Morpho is a user-friendly application designed
to facilitate the creation of metadata (information that describes your
data) so that you and others can easily locate and determine the nature
of a wide range of data sets. By specifying some basic information (a
title and abstract, for example) about your data in a uniform,
standardized way, you or any one you have granted permission to access
your data will be able to find and view the data. When you create a
metadata file that explains what your data represent and how they are
organized, you are not only better able to manage the data, you help
other scientists discover and understand them too. 

By default, Morpho interfaces with the Knowledge Network for Biocomplexity (KNB)
DataONE Member Node, which is a Metacat server from which scientists can
upload, download, store, query and view relevant metadata and data. Once
you have annotated your data with metadata, you can choose to upload
your data--or just your data description (the metadata)--to the Member 
Node, where they can be accessed from the web by selected colleagues
or by the public if you so choose. Data stored on the Member Node server
can also be automatically saved on several geographically separate servers, 
ensuring that data are archived securely throughout the DataONE network.

\subsection{Terms you need to know}

Throughout this guide, we refer to \hyperref[sec:metadata]{metadata}
and \hyperref[sec:data package]{data package}. Both terms are briefly
defined below.

\subsubsection{Metadata} \label{sec:metadata}

In Morpho, the metadata--or data describing data--contains information
about the content of a data set (its owner, administrator, geographic
extent, units, etc) as well as who has access to the data (the owner,
selected users, or the public). This information is stored in a file
that conforms to the Ecological Metadata Language (EML) specification,
which is commonly used to exchange information among scientists across
the world.

When you use one of Morpho's easy-to-use wizards to create a metadata
file, Morpho automatically takes the values you enter and generates the
metadata file in the proper format. The metadata file is stored on your
local system and/or on the DataONE network. Metadata can be "packaged" with
the data set, or can stand alone� much like an abstract describing the
contents of a paper. 

The Morpho wizards create metadata files using a subset of Ecological
Metadata Language (EML), a metadata specification developed by ecology
discipline but that has since gained wider usage. EML is based on prior
work done by the Ecological Society of America and associated efforts
(Michener et al., 1997, Ecological Applications 7: 330-342). For more
information about EML, see
\url{http://knb.ecoinformatics.org/software/eml/}


\subsubsection{Data Package} \label{sec:data package}

Data packages are the logical units that Morpho creates to represent a
collection of metadata and (optionally) data files. At its most basic, a
data package consists only of high-level documentation: metadata about a
data collection's title and abstract, keywords, people and
organizations, usage rights, research project information, coverage
details, methods and sampling, and access information. Once a basic data
package has been created, you can add metadata for the individual data
tables (row and column information) and optionally include the data
tables themselves in the package. 

Data packages can be uploaded to the DataONE network and shared with
colleagues, or stored locally on your system.

