\section{Technical Notes} \label{sec:technical-notes}

For those who are interested in technical details, the Morpho editor is
an XML editor. It works by first reading an XML file and building an
outline (tree) view of the XML document. XML files can have formalized
templates called ``DTD's'' (Document Type Definition), which describe how
the document can be constructed. If the XML document indicates that its
structure should conform to a DTD, then the DTD is scanned and a single
instance of any optional nodes not present in the original is added to
the hierarchy. Finally, if the editor has additional data about the
document type, it will add that data as custom displays or help
information about the node, as is shown in the above examples. The
editor can thus be customized to display the XML data in a variety of
ways.
